\documentclass{beamer}
\usetheme{Boadilla}
\usepackage{hyperref}
\usepackage{graphicx}
\usepackage{fancyvrb}
\usepackage{multicol}
\usepackage{subfig}
\usepackage[
    backend=biber, 
    natbib=true,
    style=numeric,
    sorting=none,
    style=verbose-ibid,
]{biblatex}
\addbibresource{citations.bib}
\usepackage{pgfpages}
\usepackage{xcolor}
\definecolor{ao(english)}{rgb}{0.0, 0.5, 0.0}
\definecolor{burgundy}{rgb}{0.5, 0.0, 0.13}
%\setbeameroption{show notes}
\setbeameroption{show notes on second screen=right}
%\setbeameroption{hide notes}

\title{The Opus Codec}
\subtitle{High-quality, low-delay music codec}
\author{Sevag Hanssian}
\date{Feburary 09, 2021}
\institute{MUMT 621, Winter 2021}
\setbeamertemplate{navigation symbols}{}

\begin{document}

\begin{frame}
\maketitle
\end{frame}

\begin{frame}
	\frametitle{Xiph.org}
	\begin{quote}
		Xiph.Org is a collection of open source, multimedia-related projects.
	\end{quote}
	\begin{enumerate}
			\vspace{-0.5em}
		\item
			FLAC, Vorbis, Opus, Speex, \textit{Daala, Theora}
		\item
			Misc: RNNoise
	\end{enumerate}
	\vspace{1em}
	\begin{quote}
		The most aggressive effort works to put the foundation standards of Internet audio and video into the public domain, where all Internet standards belong.
	\end{quote}
	\begin{enumerate}
			\vspace{-0.5em}
		\item
			Closed software and protocols are not evil or worse than open source, but by definition only exist serve the bottom line of a corporation
		\item
			 The Internet is built on open development, free exchange of ideas and intellectual cooperation
	\end{enumerate}
\end{frame}

\note{
	\begin{itemize}
		\item
			I.e. they are not in the public's best interest
		\item
			Google AMP, facebook, twitter
		\item
			Google AMP - mention technical difficulty
		\item
			This is really becoming a problem today
	\end{itemize}
}

\begin{frame}
	\frametitle{MP3 and the music industry}
	summarize xiph story
	"Why does multimedia specifically need open source?"
	https://xiph.org/about/
\end{frame}

\note{
	\begin{itemize}
		\item
			insert notes here
	\end{itemize}
}

\begin{frame}
	\frametitle{Ogg/Vorbis, containers and codecs}
	Matroska, webm, mkv
	When an Opus audio track is stored in an Ogg container, the file name suffix is usually '.opus' (but when a Vorbis audio track is stored in an Ogg file, the suffix is usually '.ogg')
\end{frame}

\begin{frame}
	\frametitle{The Opus Codec}
	Audio codec designed for the Internet
	\begin{itemize}
		\item
			Open-source, royalty-free
		\item
			Lossy
		\item
			Derives from:
			\begin{itemize}
				\item
					CELT (\textbf{C}onstrained-\textbf{Energy} and \textbf{L}apped \textbf{T}ransform)
				\item
					SILK, Skype speech codec
			\end{itemize}
		\item
			Replaces Vorbis (music codec) and Speex (speech codec) in a single codec
	\end{itemize}
	\vspace{1em}
	Opus can operate in three modes:
	\begin{enumerate}
		\item
			SILK mode for speech -- low bitrate narrowband speech
		\item
			CELT mode for music -- high bitrate, high quality music
		\item
			Hybrid -- SILK <8kHz, CELT >8kHz
	\end{enumerate}
\end{frame}

\note{
\begin{itemize}
	\item
		SILK is not an acronym
	\item
		CELT used to be a standalone algorith, now its the music part of opus
\end{itemize}
}

\begin{frame}
	\frametitle{Speech and music}
	why different codecs? general properties of sound
	narrowband
	wideband
\end{frame}

\begin{frame}
	\frametitle{Lossy and lossless codecs}
\end{frame}

\begin{frame}
	\frametitle{Psychoacoustics}
\end{frame}

\begin{frame}
	\frametitle{Opus RFCs and timeline}
	rfcs and timeline - https://jmvalin.ca/opus/opus-1.3/
\end{frame}


\begin{frame}
	\frametitle{Ambisonics, spatial features}
\end{frame}

\note{
	\begin{itemize}
		\item
			jake mentioned
	\end{itemize}
}

\end{document}
