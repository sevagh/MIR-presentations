\documentclass[letter,12pt]{report}
\setlength{\parindent}{0pt}
\usepackage[left=2cm, right=2cm, top=2cm, bottom=2cm]{geometry}
\usepackage[shortlabels]{enumitem}
\usepackage{graphicx}
\usepackage{amsmath}
\usepackage{amssymb}
\usepackage{verbatim}
\usepackage{listings}
\usepackage{minted}
\usepackage{subfig}
\usepackage{titling}
\usepackage{caption}
\setlength{\droptitle}{1cm}
\usepackage{hyperref}
\hypersetup{
    colorlinks,
    citecolor=black,
    filecolor=black,
    linkcolor=black,
    urlcolor=black
}
\usepackage{setspace}
\renewcommand{\topfraction}{0.85}
\renewcommand{\textfraction}{0.1}
\renewcommand{\floatpagefraction}{0.75}
\usepackage[
    %backend=biber, 
    natbib=true,
    style=numeric,
    sorting=none
]{biblatex}
\addbibresource{citations.bib}
\usepackage{titlesec}
 
\titleformat{\chapter}[display]
  {\normalfont\bfseries}{}{0pt}{\Huge}

\begin{document}

\Large{\textbf{Annotated bibliography -- Opus audio codec}}\\
\large{MUMT 621 Presentation 2. February 09, 2021. Sevag Hanssian, 260398537}

\begin{itemize}
	\item
		\fullcite{opuspaper}\\
		This is the published paper for the music (CELT) details of the Opus codec (and some minor information on the speech/SILK mode).
	\item
		\fullcite{xiphwebsite}\\
		Xiph.org's ``About'' page on their website, where a full list of their projects, mission statement, and origins (the 1998 MP3 situation) are outlined.
	\item
		\fullcite{valinopus}\\
		Jean-Marc Valin (one of the principal authors of Opus and members of Xiph.org) talks about the Opus 1.3 release and describes the ambisonic features and the music-speech detection neural network.
	\item
		\fullcite{larynx}\\
		Lectures slides from a McMaster's Electrical Engineering course describing speech compression and linear prediction coding in terms of the larynx, throat, glottis, excitation signal and tube coefficients.
	\item
		\fullcite{mpegwebsite}\\
		MPEG's website from where I copied their mission statement.
	\item
		\fullcite{computall}\\
		Computall, a defunct tech company, describes how theyhad to cease work on their MP3 software, and published the emails they received from Fraunhofer and Thomson re: MP3 royalties in 1998.
	\item
		\fullcite{riaaweb}\\
		RIAA's website where I copied their mission and anti-piracy statements.
	\item
		\fullcite{pitivi}\\
		Small website by Pitivi.org (open source media player) describing the difference between containers and codecs which I used to help word my explanation.
	\item
		\fullcite{riaavdiamond}\\
		A report on the court case of RIAA versus Diamond, the MP3 player manufacturer.
	\item
		\fullcite{skiena}\\
		Skiena's famous Algorithm Design manual, from which I looked up lossy and lossless compression.
	\item
		\fullcite{lossypsycho}\\
		An examination and description of lossy psychacoustic audio compression methods, which I used to describe the idea that lossy audio codecs go hand-in-hand with psychoacoustics to ensure the loss is minimally percetible.
	\item
		\fullcite{generationloss}\\
		A paper describing what generation loss is, to add context to Xiph.org's recommendation that Opus should not be used for archival due to generation loss.
	\item
		\fullcite{silk}\\
		The IETF draft RFC for the SILK audio codec, containing  many technical details.
	\item
		\fullcite{speech}\\
		A report on different speech and audio bands (narrowband, wideband) in my speech vs. music slide.
	\item
		\fullcite{moore}\\
		A study by famous psychoacoustic researcher Brian C. J. Moore, describing that hearing aids designed for speech frequencies are not so good at music (which helps me provide rationale for why Opus contains different coding strategies for speech and music).
	\item
		\fullcite{opuspapervoice}\\
		A second paper on Opus by the codec authors, this time focused on the speech/SILK part of the codec.
	\item
		\fullcite{dct}\\
		The original paper presenting the DCT, comparing it to the DFT, and describing its special property of ``spectral compactness.''
\end{itemize}

\end{document}
