\documentclass[letter,11pt]{report}
%\setlength{\parindent}{0pt}
\usepackage[left=1in, right=1in, top=1in, bottom=1in]{geometry}
\usepackage[shortlabels]{enumitem}
\usepackage{graphicx}
\usepackage{amsmath}
\usepackage{amssymb}
\usepackage{verbatim}
\usepackage{listings}
\usepackage{minted}
\usepackage{subfig}
\usepackage{titling}
\usepackage{caption}
\setlength{\droptitle}{1cm}
\usepackage{hyperref}
\hypersetup{
    colorlinks,
    citecolor=black,
    filecolor=black,
    linkcolor=black,
    urlcolor=black
}
\usepackage{setspace}
\renewcommand{\topfraction}{0.85}
\renewcommand{\textfraction}{0.1}
\renewcommand{\floatpagefraction}{0.75}
\usepackage[backend=biber,authordate,annotation]{biblatex-chicago}
\addbibresource{citations.bib}
\usepackage{titlesec}
 
\titleformat{\chapter}[display]
  {\normalfont\bfseries}{}{0pt}{\Huge}

\begin{document}

\noindent\LARGE{\textbf{Pitch Tracking}}\\
\Large{MUMT 621 Presentation 4. March 16, 2021. Sevag Hanssian, 260398537}\\
%\vspace{0.15em}

\noindent\Large{\textbf{Summary}}
%\vspace{0.15em}

Pitch is the perceptual correlate of frequency (\cite{plack}). \textcite{plack} mentions the following definition of pitch by the American National Standards Institute: ``pitch is that attribute of auditory sensation in terms of which sounds may be ordered on a scale extending from low to high.'' The problem with this definition is that it contains ambiguity -- the terms ``high'' and ``low'' can also apply to loudness and intensity. He proposes an alternative definition, that pitch is the aspect of auditory sensation whose variation is associated with musical melodies.

For a pure tone consisting of a single frequency, the pitch of the tone is directly related to its frequency. For a complex tone with multiple frequency components, the pitch is related to the fundamental, or lowest frequency. The fundamental frequency is also referred to as $\mathit{f0}$.

Pitch is essential to music perception in humans (\cite{musicevo}). Relationships between pitches are more important than the absolute pitch -- humans in experiments can recognize melodies when they are transposed in tempo or octave. Pitches separated by the special octave relationship have the same pitch chroma. Most known music in the world across cultures and eras consist of melodies formed from five to seven pitches within in octave.

According to \textcite{moore}, for years there were two theories on human pitch perception: place and temporal theories. The place theory, or place coding, is related to the tonotopic organization of the inner ear: spectral analysis is done in the cochlea, and the resolved harmonics of a sound excite different parts of the basilar membrane (BM), causing with different characteristic frequency (CF) to fire. The activated center frequencies may then be compared to a pattern -- for example, a complex tone with a fundamental frequency of 100Hz and two harmonics at 200Hz and 300Hz, is compared to a template and found to match the harmonic template of a pitch of 100Hz.

In the temporal theory, the unresolved harmonics form a complex waveform in the BM, and firing neurons lock to the phase of the envelope of the complex waveform. More complete, newer models that try to account for all the available experimental data rely on both place and temporal analyses.

Computational pitch tracking has been studied for at least half a century (\cite{nollcepstrum}).

CREPE (\cite{crepe}), the current state-of-the-art neural network for pitch tracking presents an overview of several computational pitch tracking algorithms proposed over the years. 

\vfill
\clearpage

\noindent\LARGE{\textbf{Bibliography}}\\

\vspace{-0.5em}

%\printbibheading[title={\vspace{-3.5em}References},heading=bibnumbered]
%\nocite{*}
\printbibliography[heading=none]

\end{document}
