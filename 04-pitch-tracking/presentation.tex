\documentclass{beamer}
\usetheme{Boadilla}
\usepackage{hyperref}
\usepackage{graphicx}
\usepackage{fancyvrb}
\usepackage{multicol}
\usepackage{adjustbox}
\usepackage{tikz}
\usetikzlibrary{shapes,positioning}
\newcommand{\foo}{\hspace{-2.3pt}$\bullet$ \hspace{5pt}}
\usepackage{subfig}
\usepackage[backend=biber,authordate]{biblatex-chicago}
\addbibresource{citations.bib}
\usepackage{pgfpages}
\usepackage{xcolor}
\definecolor{ao(english)}{rgb}{0.0, 0.5, 0.0}
\definecolor{burgundy}{rgb}{0.5, 0.0, 0.13}
%\setbeameroption{show notes}
\setbeameroption{show notes on second screen=right}
%\setbeameroption{hide notes}

\title{Pitch Tracking}
\author{Sevag Hanssian}
\date{March 23, 2021}
\institute{MUMT 621, Winter 2021}
\setbeamertemplate{navigation symbols}{}

\begin{document}

\begin{frame}
\maketitle
\end{frame}

\begin{frame}
	\frametitle{Pitch as a perceptual phenomenon}
	Pitch is...\footfullcite{plack}
	\begin{itemize}
		\item
			the perceptual correlate of frequency or waveform repetition rate
		\item
			the attribute of auditory sensation in terms of which sounds may be ordered on a scale extending from low to high (\textcolor{red}{\textbf{!}})
		\item
			the aspect of auditory sensation whose variation is associated with musical melodies (\textcolor{green}{\textbf{!}})
	\end{itemize}
	For sinusoidal sounds, pitch and frequency are directly related. The fundamental frequency of a complex tone (consisting of multiple frequencies) is the lowest (or fundamental) frequency component -- higher frequencies are called harmonics or overtones.

	Apart from a few rare exceptions, pitch can be quantified using fundamental frequency (or $\mathit{f0}$), and thus they are often used interchangeably outside psychoacoustical studies.\footfullcite{crepe}
\end{frame}

\note{
	\begin{itemize}
		\item
			it's a bit more complicated. there's also a relationship between loudness and perceived pitch
			It requires the words ``low'' and ``high'' to be associated with pitch or frequency rather than with loudness or intensity, for example.
	\end{itemize}
}

\begin{frame}
	\frametitle{Musical importance of pitch}
	\begin{quote}
	Although rhythm is arguably just as important, if not more so, to many cultures' music, pitch has received far more attention in the literature we will review. This is likely due to its importance in Western music and the resultant theoretical ideas about how pitch functions in music.
	\end{quote}
	Pitch is musically important to humans:\footfullcite{musicevo}
	\begin{itemize}
		\item
			Relationship between pitches (relative pitch) are more important than the absolute value. Sequence of pitch changes is the melodic contour
		\item
			Pitches separated by an octave have the same pitch chroma. Most (known) music depends on pitch relations defined by octaves
		\item
			Most (known) music in the world come from a discrete set of five to seven pitches arranged within an octave range
		\item
			Most musical scales in the world have pitches separated by unequal steps; allows assigning unique functions to different notes
	\end{itemize}
\end{frame}

\note{
	\begin{itemize}
		\item
			Melodies can be recognized when transposed. Pitch is universal, but the centrality of relative pitch indicates some innate auditory mechanism for encoding sound as pitch distances. Humans are good at ``melodic contour'', i.e. recognizing a melody whether it is transposed in tempo or octave.
		\item
			Limitations of memory and categorization, or sensory or computational bias to have intervals that approximate simple integer ratios
		\item
			An example of note function is the tonic
	\end{itemize}
}

\begin{frame}
\frametitle{Models for human pitch perception}
For many years, there were two theories of pitch perception: place and temporal\footfullcite{moore}

The place theory (place coding) applies to \textit{resolved harmonics}:
\begin{itemize}
	\item
		Spectral analysis is done in the cochlea, so that the frequency components of a sound excite different parts of the basilar membrane (BM), hence neurons with different characteristic frequency (\textcolor{green}{\textbf{!}})
	\item
		The pattern of excitation is considered for the final pitch consideration of a stimulus (\textcolor{red}{\textbf{!}})
\end{itemize}

The temporal theory (temporal coding, phase locking) applies to \textit{unresolved harmonics}: the unresolved harmonics form a complex waveform in the BM; neurons lock to the phase of the envelope of the complex waveform.

More sophisticated models (which explain or account for human experimental data) require both place and temporal analyses.
\end{frame}

\note{
	\begin{itemize}
		\item
			First part is related to the tonotopic organization of the inner ear -- well established and independently confirmed in different studies
		\item
			e.g. 100Hz, 200Hz, 300Hz pattern. Second is still a matter of dispute
		\item
			place/pattern model does not account for the fact that pitch can be recognized even with unresolved harmonics. Nonresolved harmonic = unseparated frequencies
		\item
			gabor tf uncertainty principle. Our real sensitivity to <1\% differences in fundamental frequencies for resolved harmonics is better than what would be possible with firing rate/place coding, suggesting temporal coding
		\item
			Tentative neural component, processed after the ears -- two harmonics presented to different ears create the correct fundamental.
	\end{itemize}
}

\begin{frame}
	\frametitle{Pitch tracking algorithms}
Autocorrelation
\end{frame}

\begin{frame}
	\frametitle{McLeod Pitch Method}
\end{frame}

\begin{frame}
	\frametitle{YIN}
\end{frame}

\begin{frame}
	\frametitle{pYIN}
\end{frame}

\begin{frame}
	\frametitle{CREPE}
\end{frame}

\begin{frame}
	\frametitle{Conclusion}
\end{frame}

\end{document}
